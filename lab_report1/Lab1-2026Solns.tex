% Options for packages loaded elsewhere
\PassOptionsToPackage{unicode}{hyperref}
\PassOptionsToPackage{hyphens}{url}
\PassOptionsToPackage{dvipsnames,svgnames,x11names}{xcolor}
%
\documentclass[
  12 pt,
]{article}
\usepackage{amsmath,amssymb}
\usepackage{iftex}
\ifPDFTeX
  \usepackage[T1]{fontenc}
  \usepackage[utf8]{inputenc}
  \usepackage{textcomp} % provide euro and other symbols
\else % if luatex or xetex
  \usepackage{unicode-math} % this also loads fontspec
  \defaultfontfeatures{Scale=MatchLowercase}
  \defaultfontfeatures[\rmfamily]{Ligatures=TeX,Scale=1}
\fi
\usepackage{lmodern}
\ifPDFTeX\else
  % xetex/luatex font selection
\fi
% Use upquote if available, for straight quotes in verbatim environments
\IfFileExists{upquote.sty}{\usepackage{upquote}}{}
\IfFileExists{microtype.sty}{% use microtype if available
  \usepackage[]{microtype}
  \UseMicrotypeSet[protrusion]{basicmath} % disable protrusion for tt fonts
}{}
\makeatletter
\@ifundefined{KOMAClassName}{% if non-KOMA class
  \IfFileExists{parskip.sty}{%
    \usepackage{parskip}
  }{% else
    \setlength{\parindent}{0pt}
    \setlength{\parskip}{6pt plus 2pt minus 1pt}}
}{% if KOMA class
  \KOMAoptions{parskip=half}}
\makeatother
\usepackage{xcolor}
\usepackage[margin=1in]{geometry}
\usepackage{graphicx}
\makeatletter
\newsavebox\pandoc@box
\newcommand*\pandocbounded[1]{% scales image to fit in text height/width
  \sbox\pandoc@box{#1}%
  \Gscale@div\@tempa{\textheight}{\dimexpr\ht\pandoc@box+\dp\pandoc@box\relax}%
  \Gscale@div\@tempb{\linewidth}{\wd\pandoc@box}%
  \ifdim\@tempb\p@<\@tempa\p@\let\@tempa\@tempb\fi% select the smaller of both
  \ifdim\@tempa\p@<\p@\scalebox{\@tempa}{\usebox\pandoc@box}%
  \else\usebox{\pandoc@box}%
  \fi%
}
% Set default figure placement to htbp
\def\fps@figure{htbp}
\makeatother
\setlength{\emergencystretch}{3em} % prevent overfull lines
\providecommand{\tightlist}{%
  \setlength{\itemsep}{0pt}\setlength{\parskip}{0pt}}
\setcounter{secnumdepth}{-\maxdimen} % remove section numbering
\usepackage{bookmark}
\IfFileExists{xurl.sty}{\usepackage{xurl}}{} % add URL line breaks if available
\urlstyle{same}
\hypersetup{
  pdftitle={ST231 Lab Report 1},
  colorlinks=true,
  linkcolor={Maroon},
  filecolor={Maroon},
  citecolor={Blue},
  urlcolor={blue},
  pdfcreator={LaTeX via pandoc}}

\title{ST231 Lab Report 1}
\author{}
\date{\vspace{-2.5em}Deadline: 3 February 2026, 1 pm}

\begin{document}
\maketitle

\section{Data Description}\label{data-description}

The data in the file \texttt{dia1.csv} is an adapted subset from the
\texttt{diamond}s dataset in the \texttt{ggplot2} package. It consists
of information on 800 round diamonds. The variables considered for this
lab report are:

\begin{itemize}
\tightlist
\item
  \texttt{price}: the sale price of the diamond in US Dollars.
\item
  \texttt{weight}: the weight of the diamond in carat.
\end{itemize}

\section{Instructions}\label{instructions}

\textbf{Carefully read the
\href{https://moodle.warwick.ac.uk/mod/page/view.php?id=2767925}{Lab
Report Guidance} on moodle before you start working on this lab report!}

In the following you will explore the relationship between the weight of
a diamond and its price.

\begin{enumerate}
\def\labelenumi{\arabic{enumi}.}
\item
  \textbf{{[}2 marks{]}} Produce a scatterplot that illustrates the
  relationship between the weight of a diamond and its price. Fit a
  quadratic and a cubic regression model to the data with \texttt{price}
  as the response variable and \texttt{weight} as the explanatory
  variable. Add the fitted curves of the two polynomial models to the
  scatterplot.
\item
  \textbf{{[}3 marks{]}} Based on the evidence from the plot in Question
  1, critically evaluate and compare the fit of the two models to the
  data.
\item
  \textbf{{[}3 marks{]}} For each of the two fitted models, produce a
  residual plot, that is a plot of residuals against fitted values.
  Critically evaluate and compare the two plots.
\end{enumerate}

\textbf{{[}2 marks{]}} for style and quality of presentation.

\newpage

\textbf{Feedback:}

\begin{enumerate}
\def\labelenumi{\arabic{enumi}.}
\tightlist
\item
  R markdown syntax and cheat sheet.
\item
  Hide code blocks, \texttt{echo=FALSE}.
\item
  \texttt{fig.cap}
\item
  Question 2 and 3
\end{enumerate}

\textbf{Solutions:}

\begin{enumerate}
\def\labelenumi{\arabic{enumi}.}
\tightlist
\item
  The scatterplot below shows the data and the fitted curves for the
  quadratic regression model (blue solid line) and the cubic regression
  model (dashed red line) which both predict the price of a diamond from
  its weight.
\end{enumerate}

\begin{figure}

{\centering \includegraphics[width=0.8\linewidth,alt={A scatterplot of diamond prices in Dollars against the weight of the diamond in carat. The fitted curves for a quadratic regression model and a cubic regression model have been added to the plot.  For further comments on the plot, see the main text.}]{Lab1-2026Solns_files/figure-latex/Question2-1} 

}

\caption{A scatterplot of diamond prices in Dollars against the weight of the diamond in carat. The fitted curves for a quadratic regression model and a cubic regression model have been added to the plot.}\label{fig:Question2}
\end{figure}

\begin{enumerate}
\def\labelenumi{\arabic{enumi}.}
\setcounter{enumi}{1}
\item
  Both models take account of the fact that the relationship between the
  weight of diamonds and their price is curved rather than linear.
  However the quadratic model systematically underestimates prices for
  diamonds that have a small weight, even predicting negative prices. In
  contrast the cubic model appears to fit the data more closely. In
  particular, it avoids predicting negative prices for lighter diamonds.
  However, it shows some artifacts that come from the properties of a
  cubic function. The fitted cubic curve attains a maximum at around
  2.25 carat and then decreases which does not fit with the general
  observation that diamonds tend to be more expensive the heavier they
  are. (We observe a similar artifact weights close to zero where the
  curve attains a minimum.)

  \textbf{Note:} In this specific example the artifacts of the cubic
  model occur at the boundary of the range of the data where there are
  few observations to infer the shape of the relationship between weight
  and price. Here we would always be cautious.
\end{enumerate}

\newpage

\begin{enumerate}
\def\labelenumi{\arabic{enumi}.}
\setcounter{enumi}{3}
\item
  The residual plot for the quadratic regression model indicates some
  mild non-linearity as the smoother first decreases and then increases.
  The non-linearity is less pronounced in the residual plot for the
  cubic regression model as the smoother is initially flat and only
  increases for larger fitted values. In both plots, the observations
  form a right-opening megaphone pattern, which indicates
  hetero-scedasticity with a variance that is increasing with fitted
  value.

  \textbf{Note:} When we fit polynomial models, we always include any
  lower order terms. For example, in the cubic modoel we included a
  cubic term \texttt{I(weight\^{}3)}, but also a quadratic term
  \texttt{I(weight\^{}2)} and a linear term \texttt{weight}.
\end{enumerate}

\begin{figure}

{\centering \includegraphics[width=0.48\linewidth,alt={Residual versus fitted values plots. Left: residual plot for the quadratic regression model. Right:  residual plot for the cubic regression model. For further comments on the plot, see the main text.}]{Lab1-2026Solns_files/figure-latex/Question3-1} \includegraphics[width=0.48\linewidth,alt={Residual versus fitted values plots. Left: residual plot for the quadratic regression model. Right:  residual plot for the cubic regression model. For further comments on the plot, see the main text.}]{Lab1-2026Solns_files/figure-latex/Question3-2} 

}

\caption{Residual versus fitted values plots. Left: residual plot for the quadratic regression model. Right:  residual plot for the cubic regression model.}\label{fig:Question3}
\end{figure}

\end{document}
