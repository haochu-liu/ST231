% Options for packages loaded elsewhere
\PassOptionsToPackage{unicode}{hyperref}
\PassOptionsToPackage{hyphens}{url}
%
\documentclass[
  12pt,
]{article}
\usepackage{amsmath,amssymb}
\usepackage{iftex}
\ifPDFTeX
  \usepackage[T1]{fontenc}
  \usepackage[utf8]{inputenc}
  \usepackage{textcomp} % provide euro and other symbols
\else % if luatex or xetex
  \usepackage{unicode-math} % this also loads fontspec
  \defaultfontfeatures{Scale=MatchLowercase}
  \defaultfontfeatures[\rmfamily]{Ligatures=TeX,Scale=1}
\fi
\usepackage{lmodern}
\ifPDFTeX\else
  % xetex/luatex font selection
\fi
% Use upquote if available, for straight quotes in verbatim environments
\IfFileExists{upquote.sty}{\usepackage{upquote}}{}
\IfFileExists{microtype.sty}{% use microtype if available
  \usepackage[]{microtype}
  \UseMicrotypeSet[protrusion]{basicmath} % disable protrusion for tt fonts
}{}
\makeatletter
\@ifundefined{KOMAClassName}{% if non-KOMA class
  \IfFileExists{parskip.sty}{%
    \usepackage{parskip}
  }{% else
    \setlength{\parindent}{0pt}
    \setlength{\parskip}{6pt plus 2pt minus 1pt}}
}{% if KOMA class
  \KOMAoptions{parskip=half}}
\makeatother
\usepackage{xcolor}
\usepackage[margin=1in]{geometry}
\usepackage{color}
\usepackage{fancyvrb}
\newcommand{\VerbBar}{|}
\newcommand{\VERB}{\Verb[commandchars=\\\{\}]}
\DefineVerbatimEnvironment{Highlighting}{Verbatim}{commandchars=\\\{\}}
% Add ',fontsize=\small' for more characters per line
\newenvironment{Shaded}{}{}
\newcommand{\AlertTok}[1]{\textbf{#1}}
\newcommand{\AnnotationTok}[1]{\textit{#1}}
\newcommand{\AttributeTok}[1]{#1}
\newcommand{\BaseNTok}[1]{#1}
\newcommand{\BuiltInTok}[1]{#1}
\newcommand{\CharTok}[1]{#1}
\newcommand{\CommentTok}[1]{\textit{#1}}
\newcommand{\CommentVarTok}[1]{\textit{#1}}
\newcommand{\ConstantTok}[1]{#1}
\newcommand{\ControlFlowTok}[1]{\textbf{#1}}
\newcommand{\DataTypeTok}[1]{\underline{#1}}
\newcommand{\DecValTok}[1]{#1}
\newcommand{\DocumentationTok}[1]{\textit{#1}}
\newcommand{\ErrorTok}[1]{\textbf{#1}}
\newcommand{\ExtensionTok}[1]{#1}
\newcommand{\FloatTok}[1]{#1}
\newcommand{\FunctionTok}[1]{#1}
\newcommand{\ImportTok}[1]{#1}
\newcommand{\InformationTok}[1]{\textit{#1}}
\newcommand{\KeywordTok}[1]{\textbf{#1}}
\newcommand{\NormalTok}[1]{#1}
\newcommand{\OperatorTok}[1]{#1}
\newcommand{\OtherTok}[1]{#1}
\newcommand{\PreprocessorTok}[1]{\textbf{#1}}
\newcommand{\RegionMarkerTok}[1]{#1}
\newcommand{\SpecialCharTok}[1]{#1}
\newcommand{\SpecialStringTok}[1]{#1}
\newcommand{\StringTok}[1]{#1}
\newcommand{\VariableTok}[1]{#1}
\newcommand{\VerbatimStringTok}[1]{#1}
\newcommand{\WarningTok}[1]{\textit{#1}}
\usepackage{graphicx}
\makeatletter
\newsavebox\pandoc@box
\newcommand*\pandocbounded[1]{% scales image to fit in text height/width
  \sbox\pandoc@box{#1}%
  \Gscale@div\@tempa{\textheight}{\dimexpr\ht\pandoc@box+\dp\pandoc@box\relax}%
  \Gscale@div\@tempb{\linewidth}{\wd\pandoc@box}%
  \ifdim\@tempb\p@<\@tempa\p@\let\@tempa\@tempb\fi% select the smaller of both
  \ifdim\@tempa\p@<\p@\scalebox{\@tempa}{\usebox\pandoc@box}%
  \else\usebox{\pandoc@box}%
  \fi%
}
% Set default figure placement to htbp
\def\fps@figure{htbp}
\makeatother
\setlength{\emergencystretch}{3em} % prevent overfull lines
\providecommand{\tightlist}{%
  \setlength{\itemsep}{0pt}\setlength{\parskip}{0pt}}
\setcounter{secnumdepth}{-\maxdimen} % remove section numbering
\usepackage{bookmark}
\IfFileExists{xurl.sty}{\usepackage{xurl}}{} % add URL line breaks if available
\urlstyle{same}
\hypersetup{
  pdftitle={Practical Report 2},
  hidelinks,
  pdfcreator={LaTeX via pandoc}}

\title{Practical Report 2}
\author{}
\date{\vspace{-2.5em}}

\begin{document}
\maketitle

\section{Acceptable residual plots}\label{acceptable-residual-plots}

In this practical we will explore simulated data to illustrate residual
diagnostics which are part of the linear modelling process. Residual
plots allows us to judge the appropriateness of the assumption of
linearity and homoscedasticity.

\subsection{Section 1, Question 1}\label{section-1-question-1}

Linear model:
\[Y_j \quad = \quad  3 + \frac{1}{2}x_j + \epsilon_j, \qquad j=1,\ldots, 100,\]
where \(\epsilon_1, \ldots, \epsilon_{100}\) are iid \(N(0, 4)\).
(Remark: \texttt{rnorm} takes standard deviation as the argument, see in
\texttt{?rnorm}.)

\subsection{Section 1, Question 2}\label{section-1-question-2}

\begin{figure}

{\centering \includegraphics[width=0.7\linewidth]{Practical2Report_files/figure-latex/SLR1-1} 

}

\end{figure}

\begin{verbatim}
## (Intercept)           x 
##   2.6376093   0.5052425
\end{verbatim}

We can see that the estimated intercept and slope are reasonably close
to the true parameter, 3 and 0.5. We can double-check this by the 95\%
confidence intervals.

\begin{verbatim}
##                 2.5 %    97.5 %
## (Intercept) 1.8305475 3.4446711
## x           0.4913678 0.5191172
\end{verbatim}

\subsection{Section 1, Question 3}\label{section-1-question-3}

\begin{figure}

{\centering \includegraphics[width=0.7\linewidth]{Practical2Report_files/figure-latex/Residual1-1} 

}

\end{figure}

The residuals scatter around the zero horizontal line and the variation
of the residuals appears relatively constant. This implies no evidence
of violations against the assumptions.

\subsection{Section 1 Question 4}\label{section-1-question-4}

Next, we sample Dataset 2 with only 20 data points.

\begin{figure}

{\centering \includegraphics[width=0.49\linewidth]{Practical2Report_files/figure-latex/LinearModel2-1} \includegraphics[width=0.49\linewidth]{Practical2Report_files/figure-latex/LinearModel2-2} 

}

\caption{Left: Right:}\label{fig:LinearModel2}
\end{figure}

Harder to tell if satisfying the model assumptions. We can see some
curvature and uneven variance of the residuals even if the data is
sampled from a linear model.

Question to think: how the confidence intervals change?

\begin{verbatim}
##                 2.5 %    97.5 %
## (Intercept) 2.3082453 4.1203836
## x2          0.3753934 0.5266678
\end{verbatim}

\begin{verbatim}
## [1] "model2 intercept CI length: 1.81213825014775"
\end{verbatim}

\begin{verbatim}
## [1] "model2 slop CI length: 0.151274378348051"
\end{verbatim}

\begin{verbatim}
## [1] "model1 intercept CI length: 1.61412358684011"
\end{verbatim}

\begin{verbatim}
## [1] "model1 slop CI length: 0.0277494061464172"
\end{verbatim}

\section{Unacceptable residual plots}\label{unacceptable-residual-plots}

The figure below shows the scatterplot of Dataset 3, a dataset sampled
from the model \[ Y_j \quad = \quad 10 + \frac{1}{2}x_j + \epsilon_j,\]
where \(x_j = j\) and \(\epsilon_j \sim N(0, x_j)\) for
\(j=1, \ldots, 100\). Furthermore, the errors
\(\epsilon_1, \ldots, \epsilon_{100}\) are independent. We fitted a
simple linear regression model to the data and added the fitted line to
the scatterplot.

\subsection{Section 2 Question 1}\label{section-2-question-1}

\begin{figure}

{\centering \includegraphics[width=0.7\linewidth]{Practical2Report_files/figure-latex/data3-1} 

}

\end{figure}

The residuals scatter around the zero horizontal line. \textbf{But} the
variation of the residuals disperses as the fitted values increasing.

\begin{figure}

{\centering \includegraphics[width=0.7\linewidth]{Practical2Report_files/figure-latex/ResidualPlot1-1} 

}

\end{figure}

\subsection{Section 1, Question 2}\label{section-1-question-2-1}

\begin{figure}

{\centering \includegraphics[width=0.7\linewidth]{Practical2Report_files/figure-latex/model3_sqrt-1} 

}

\end{figure}
\begin{figure}

{\centering \includegraphics[width=0.7\linewidth]{Practical2Report_files/figure-latex/model3_sqrt-2} 

}

\end{figure}

\begin{figure}

{\centering \includegraphics[width=0.7\linewidth]{Practical2Report_files/figure-latex/model3_log-1} 

}

\end{figure}
\begin{figure}

{\centering \includegraphics[width=0.7\linewidth]{Practical2Report_files/figure-latex/model3_log-2} 

}

\end{figure}

\textbf{Comments: Linearity is a more important assumption than
homoscedasticity. Thus, if we cannot resolve the heteroscedasticity
without the linearity assumption becoming unreasonable, then we choose a
model for which the linearity assumption is reasonable, even if it
suffers from heteroscedasticity.}

\subsection{Section 2 Question 3}\label{section-2-question-3}

Our final artificial dataset is sampled from the model
\[ Y_j \quad = \quad 100 - 20 x_j  + x_j^2 + \epsilon_j,\] where
\(\epsilon_1, \ldots, \epsilon_{100}\) are iid \(N(0, 400)\).

\begin{figure}

{\centering \includegraphics[width=0.7\linewidth]{Practical2Report_files/figure-latex/data4-1} 

}

\caption{Scatterplot of Dataset 4.}\label{fig:data4}
\end{figure}

\subsection{Section 2, Question 4}\label{section-2-question-4}

\[ Y_j \quad = \quad \beta_0 + \beta_1 x_{j} + \epsilon_j. \]

\begin{center}\includegraphics[width=0.7\linewidth]{Practical2Report_files/figure-latex/linear_model4-1} \end{center}

\begin{center}\includegraphics[width=0.7\linewidth]{Practical2Report_files/figure-latex/linear_model4-2} \end{center}

\[ Y_j \quad = \quad \beta_0 + \beta_1 x_{j} + \beta_2\, x_{j}^2 + \epsilon_j. \]

\begin{center}\includegraphics[width=0.7\linewidth]{Practical2Report_files/figure-latex/quadratic_model4-1} \end{center}

Only the residual plot of the quadratic regression model is a null plot.

\subsection{Section 2, Question 5 and
6}\label{section-2-question-5-and-6}

We make use of the function \texttt{quadraticPlot} to visualise the
quadratic regression model. For element \texttt{i} in the \texttt{ax}
vector, \texttt{predict(m,\ newdata=list(x=ax))} provides
\(\hat \beta_0 + \hat \beta_1 ax[i] + \hat \beta_2 ax[i]^2\).

\begin{Shaded}
\begin{Highlighting}[]
\NormalTok{quadraticPlot }\OtherTok{\textless{}{-}} \ControlFlowTok{function}\NormalTok{(x, y)\{}
  \CommentTok{\# Produce a scatterplot of the data}
  \FunctionTok{plot}\NormalTok{(x, y, }\AttributeTok{xlab=}\StringTok{"x"}\NormalTok{, }\AttributeTok{ylab=}\StringTok{"y"}\NormalTok{, }\AttributeTok{main=}\StringTok{"Scatterplot"}\NormalTok{)}
  \CommentTok{\# Fit a quadratic regression model}
\NormalTok{  m }\OtherTok{\textless{}{-}} \FunctionTok{lm}\NormalTok{(y }\SpecialCharTok{\textasciitilde{}}\NormalTok{ x }\SpecialCharTok{+} \FunctionTok{I}\NormalTok{(x}\SpecialCharTok{\^{}}\DecValTok{2}\NormalTok{))}
  \CommentTok{\# Create a vector ax of length 101 }
  \CommentTok{\# spanning the range of the explanatory variable}
\NormalTok{  ax }\OtherTok{\textless{}{-}} \FunctionTok{seq}\NormalTok{(}\FunctionTok{min}\NormalTok{(x), }\FunctionTok{max}\NormalTok{(x), }\AttributeTok{length.out=}\DecValTok{101}\NormalTok{)}
  \CommentTok{\# predict the response for each value in ax}
\NormalTok{  fitted.curve }\OtherTok{\textless{}{-}} \FunctionTok{predict}\NormalTok{(m, }\AttributeTok{newdata=}\FunctionTok{list}\NormalTok{(}\AttributeTok{x=}\NormalTok{ax))}
  \CommentTok{\# Fit a curve through the predicted response values and }
  \CommentTok{\# add to this to the plot}
  \FunctionTok{lines}\NormalTok{(ax, fitted.curve, }\AttributeTok{col=}\StringTok{"navy"}\NormalTok{, }\AttributeTok{lwd=}\DecValTok{2}\NormalTok{) }
\NormalTok{\}}
\end{Highlighting}
\end{Shaded}

\begin{figure}

{\centering \includegraphics[width=0.7\linewidth]{Practical2Report_files/figure-latex/data3quadratic-1} 

}

\end{figure}

This scatterplot shows that the quadratic regression model provides a
good fit to the data.

\subsection{Section 2, Question 7}\label{section-2-question-7}

Consider
\[ Y_j \quad = \quad \beta_0 + \beta_1 x_{j} + \beta_2\, x_{j}^2 + \epsilon_j, \qquad j = 1, \ldots, n. \]
One unit change of \(x_j\) will affect both \(\beta_1\) and \(\beta_2\).
Hence \(\beta_1\) and \(\beta_2\) should not be interpreted
individually.

\begin{verbatim}
## (Intercept)           x      I(x^2) 
##   106.92425   -21.33701     1.04922
\end{verbatim}

\end{document}
